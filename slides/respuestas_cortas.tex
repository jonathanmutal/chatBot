\documentclass{beamer}
%
% Choose how your presentation looks.
%
% For more themes, color themes and font themes, see:
% http://deic.uab.es/~iblanes/beamer_gallery/index_by_theme.html
%
\mode<presentation>
{
  \usetheme{default}      % or try Darmstadt, Madrid, Warsaw, ...
  \usecolortheme{default} % or try albatross, beaver, crane, ...
  \usefonttheme{default}  % or try serif, structurebold, ...
  \setbeamertemplate{navigation symbols}{}
  \setbeamertemplate{caption}[numbered]
} 

\usepackage[english]{babel}
\usepackage[utf8x]{inputenc}

\title[Respuestas cortas]{Sugerencia de Respuestas Cortas para Conversaciones de Chat}
\author{Daniela Bosch, Jonathan David Mutal}
\date{Noviembre 2017}

\begin{document}

\begin{frame}
  \titlepage
\end{frame}

% Uncomment these lines for an automatically generated outline.
%\begin{frame}{Outline}
%  \tableofcontents
%\end{frame}

\section{Introduction}
\subsection{Objetivo}
\begin{frame}{Objetivo}
	\begin{block}{¿Qué?}
		\begin{itemize}
  			\item Predecir \texttt{respuestas cortas} de acuerdo al contexto.
		\end{itemize}
	\end{block}
	\begin{block}{¿Para qué?}
		\begin{itemize}
  			\item Facilitar al usuario una respuesta \texttt{inmediata}.
		\end{itemize}
	\end{block}
\end{frame}

\begin{frame}{Etapas a seguir}
	\begin{itemize}
		\item Recolección de conversaciones de chats en español.
        \item Preprocesamiento de los chats
        \item Separar el contexto de las respuestas cortas para texto de entrenamiento.
        \item Evaluar con diferentes clasificadores.
	\end{itemize}
\end{frame}


\section{Problemas}
\begin{frame}{}
	\begin{block}{Extracción de corpus}
    	Problema:
        \begin{itemize}
        	\item No hay corpus en la web.
            \item Muchos chats son privados (privacidad).
        \end{itemize}
        \pause
    	Solución:
    	\begin{itemize}
    		\item No sólo recolectar chats privados si no grupos de diferentes IM (whatsapp, telegram, etc..).
            \item Mezclar con algún corpus parecido (sms, etc..).
    	\end{itemize}
        \pause
        Alrededor de 10 personas nos han brindado sus chats...
        \begin{table}[htbp]
    		\begin{center}
        		\begin{tabular}{|l|l|l|}
        			\hline
        			IM & Tamaño & Vocab \\
        			\hline \hline
        			Whatsapp & 12MB & 59,630\\ \hline
        			Facebook & 15MB & non\\ \hline
        			Telegram & 4MB  & non\\ \hline
        		\end{tabular}
        	\caption{Tamaño de los datos de entrenamiento}
       		\end{center}
    	\end{table}
    \end{block}
\end{frame}

\begin{frame}{Datos de entrenamientos variados}
    Problemas:
	\begin{itemize}
	    \item Mezcla de dominios (conversaciones grupales y personales).
      	\item Palabras mal escritas.
        \item Palabras personalizadas.
        \item Chats con multimedia que forman parte de la conversación (imagenes, audios, videos, gif).
        \item Emoticones.
	\end{itemize}
\end{frame}


\begin{frame}{Datos de entrenamiento variados}
	Posibles soluciones:
    \begin{itemize}
      	\item Realizar un pre-procesamiento.
        \pause
        \begin{itemize}
          	 \item Eliminar STOP WORDS.
             \item Normalizar algunas palabras. Ejemplo: sisi por si. okiis por ok.
             \item Mapear cada emoticon con una palabra.
        \end{itemize}
        \pause
        \item Ignorar multimedia (por ahora).
        \item Dividir el texto de entrenamiento en diferentes dominios.
     \end{itemize}
\end{frame}

\begin{frame}{Llego el gran problema...}
	\begin{block}{¿Como usamos estos datos para hacer el bot?}
    	\pause
        Usar clasificadores (supervisado):
    	\begin{itemize}
            \item Definir una entrada (el contexto).
        	\item Definir las clases (respuestas cortas).
        \end{itemize}
    \end{block}
\end{frame}

\begin{frame}
	\begin{block}{Pero.. ¿Qué es una respuesta corta?}
		\begin{itemize}
        	\item Definimos una respuesta corta como cualquier mensaje compuesto por N tokens.
        \end{itemize}
	\end{block}
    \pause
	\begin{block}{Y.. ¿Qué es el contexto de una respuesta corta?}
    	Tenemos varias alternativas
    	\begin{itemize}
    		\item M turnos anteriores, es decir M mensajes anteriores con largo mayor igual a N. 
            \item Turnos anteriores hasta encontrar una respuesta del mismo usuario.
            \item Turnos anteriores de acuerdo al tiempo de respuesta. % Explicar un poco
   		\end{itemize}
	\end{block}
\end{frame}

\begin{frame}{Ejemplo con 3 turnos anteriores y respuesta corta de 1}
	\begin{block}{Chat}
		A: Hola B, como estas? \\
    	B: Bien y vos? \\
    	A: Super, hacemos algo hoy? \\
    	B: Dale \\
    \end{block}
    \begin{block}{Conjunto de entrenamiento}
    	Por lo que un dato de entrenamiento sería: \\
   		X: hola B, como estas? bien y vos? super, hacemos algo hoy? \\
    	Y: Dale \\
    \end{block}
\end{frame}

\begin{frame}{Caracterizando el contexto}
	Una vez definido el contexto y las clases. ¿Como caracterizamos el contexto?
    \pause
	\begin{block}{}
        \begin{itemize}
    		\item Bolsa de palabras:
            \begin{itemize}
            	\item Con palabras. Problema con nuevos ejemplos. ¿Que pasa con una palabra nueva? ¿Que pasa si hay errores de ortografía?
                \pause
                \item Con N-gramas. Seguimos teniendo el mismo problema
                \pause
                \item Con subwords. Un poquito mejor.
            \end{itemize}
            \pause
         	\item Word embeddings de diferentes dominios:
         	\begin{itemize}
            	\item SBWCE
            	\item Corpus del chat
                \item Con corpus de twitter
         	\end{itemize}
		\end{itemize}
	\end{block}
\end{frame}

\begin{frame}{Clases}
	\begin{block}{Problema}
     	Gran cantidad de clases (respuestas cortas). Imposible predecir con tan poco corpus.
    \end{block}
    \begin{block}{Solución}
   		Reducir a 3 clases. ¿Como hacerlo?
        \begin{itemize}
			\item Clustering.
            \item Respuestas más vistas en el corpus.
            \item A ojo.
		\end{itemize}
    \end{block}
\end{frame}

\begin{frame}{Clasificadores}
	\begin{block}{}
    	Algunos clasificadores a utilizar:
		\begin{itemize}
			\item SVM
            \item Logistic Regression
            \item Decision Trees
            \item Naïve Bayes
		\end{itemize}
	\end{block}
\end{frame}

\begin{frame}{Otras soluciones - Trabajo futuro}
  \begin{itemize}
    \item Definir previamente las respuestas cortas y etiquetar el corpus.
    \item Predecir varias respuestas cortas
    \item Usar un modelo de lenguaje.
    \item Usar como contexto todos los turnos anteriores pesados por proximidad 
    \item Hacer modelos neuronales seq2seq.
  \end{itemize}
\end{frame}

\begin{frame}{Colaboración}
	\begin{block}{}
		\begin{center}
       		Nosotros no pedimos monedas... solo sus chats íntimos. \\
        \end{center}
        \begin{center}
        	Agradecemos cualquier colaboración
        \end{center}
        \begin{center}
        	\{jonathanmutal95, danielarbosch\}@gmail.com \\        
        \end{center}
    \end{block}
    \begin{center}	
		Gracias por su atención!
    \end{center}
\end{frame}


\end{document}
